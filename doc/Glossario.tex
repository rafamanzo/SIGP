\documentclass[11pt, a4paper]{article}

\usepackage[brazil]{babel}
\usepackage[utf8]{inputenc}
\usepackage[T1]{fontenc}
\usepackage[pdftex]{hyperref}
\usepackage{graphicx}
\usepackage{amsmath}
\usepackage{indentfirst}
\usepackage{fancyhdr}

% Formatação
\topmargin -1.5cm
\oddsidemargin -0.04cm
\evensidemargin -0.04cm
\textwidth 16.59cm
\textheight 21.94cm 
%\pagestyle{empty}                     % Sem numero de paginas
\pagestyle{fancy}                         %cabecalhos e rodares
\fancyhead[RO,RE]{\today}
\fancyhead[LO,LE]{MAC0332 - SI para grupos de pesquisa\\
	Glossario}
\fancyfoot[LO,LE]{Confidential}
\fancyfoot[RO,RE] {\thepage}
\fancyfoot[CO,CE]{Grupo 3, 2011}


\parskip 7.2pt                        % Espaço entre paragrafos 7.2
%\renewcommand{\baselinestretch}{1.5} % Espaçamento entre linhas = 1.5
%\parindent 0pt

% Tirar hifenização
\hyphenpenalty = 5000
\tolerance = 1000
\sloppy

\title{MAC 0332\\
	Engenharia de Software\\
	SI para grupos de pesquisa\\
	Glossario}
\date{\today}
\author{Grupo(nomes e NUSP?)}

\begin{document}

	\maketitle
	\newpage

	\noindent\textbf{\huge{A}}\\
	\line(1,0){450}\\
	\textbf{Administrador}: Um membro especial do grupo de pesquisa, o único 
	que pode inscrever um novo grupo.\\
	\textbf{Annotacion}: Definições especiais para variaveis e 
	classes em Java.\\
	\textbf{Arcabouço}: Framework.\\
	
	\noindent\textbf{\huge{B}}\\
	\line(1,0){450}\\
	\textbf{Banco de Dados}: Repositorio de dados gravado em disco.\\

	\noindent\textbf{\huge{C}}\\
	\line(1,0){450}\\
	\textbf{CSS}: Cascading Style Sheets - Tecnologia de criptografia usada 
	para formatar documentos HTML, XML e XHTML.\\

	\noindent\textbf{\huge{D}}\\
	\line(1,0){450}\\
	\textbf{DER}: Diagrama Entidade Relacionamenro.\\
	\textbf{Departamento}: Contem todos os grupos de pesquisa.\\
	\textbf{Disciplinas}: Disciplinas ministradas pelo grupos.\\
	\textbf{DRY}: Don't Repeat Yourseft.\\
	
	\noindent\textbf{\huge{E}}\\
	\line(1,0){450}\\
	\textbf{Eclipse}: IDE para desenvolvimento.\\
	\textbf{Entidade}: Tabela do Banco de Dados.\\
	
	\noindent\textbf{\huge{F}}\\
	\line(1,0){450}\\
	\textbf{Framework}: Arcabouço.\\

	\noindent\textbf{\huge{G}}\\
	\line(1,0){450}\\
	\textbf{Grupo de pesquisa}: Area especifica de uma certa pesquisa, so o 
	administrador pode cria-los.\\
	
	\noindent\textbf{\huge{H}}\\
	\line(1,0){450}\\
	\textbf{Hibernate}: Promove a comunição entre o programa em Java e o Banco 
	de Dados.\\
	\textbf{HTML}: HyperText Markup Language - é uma linguagem de marcação 
	utilizada para produzir páginas na Web.\\
	
	\noindent\textbf{\huge{I}}\\
	\line(1,0){450}\\
	\textbf{IDE}: Integrated Development Environment, um ambiente integrado 
	para desenvolvimento de software.\\
	
	\noindent\textbf{\huge{J}}\\
	\line(1,0){450}\\
	\textbf{Java}: Linguagem de programação.\\
	\textbf{JSP}: JavaServer Pages - é uma tecnologia utilizada no 
	desenvolvimento de aplicações para Web em Java.\\
	\textbf{JUnit}: Testes de Unidade em Java.\\
	
	\noindent\textbf{\huge{K}}\\
	\line(1,0){450}\\
		
	\noindent\textbf{\huge{L}}\\
	\line(1,0){450}\\
	\textbf{Linhas de Pesquisa}: Pesquisas dentro de um grupo.\\
		
	\noindent\textbf{\huge{M}}\\
	\line(1,0){450}\\
	\textbf{Membros}: Membros podem colocar nova pesquisas e linhas de pesquisas, 
	o administrador tambem é um membro.\\
	\textbf{MER}: Modelo Entidade Relacionamento.\\
	\textbf{Mockito}: Framework para testes.\\
	\textbf{MVC}: Model-View-Controller - é um padrão de arquitetura de software
	que visa a separar a lógica de negócio da lógica de apresentação, permitindo
	o desenvolvimento, teste e manutenção isolado de ambos.\\
	\textbf{MySQL}: Banco de Dados relacional de codigo aberto.\\
			
	\noindent\textbf{\huge{N}}\\
	\line(1,0){450}\\
	
	\noindent\textbf{\huge{O}}\\
	\line(1,0){450}\\
	\textbf{OpenUp}: Framework de processos.\\
	\newpage
	
	\noindent\textbf{\huge{P}}\\
	\line(1,0){450}\\
	\textbf{Pencil}: Aplicativo para desenho de telas.\\
	\textbf{Projetos}: Projetos especificos do grupo de pesquisa.\\
	\textbf{Publicação}: Publicações especificas de um grupo de pesquisa.\\
	\textbf{Publico}: Todos que acessam o site podem ver.\\
	
	\noindent\textbf{\huge{Q}}\\
	\line(1,0){450}\\
	\textbf{Query}: Consulta ao Banco de Dados.\\
	
	\noindent\textbf{\huge{R}}\\
	\line(1,0){450}\\
	
	\noindent\textbf{\huge{S}}\\
	\line(1,0){450}\\
	\textbf{Subgrupo de pesquisa}: Area mais especifica que a de um grupo de 
	pesquisa, somente o admininstrador pode cria-los.\\
		
	\noindent\textbf{\huge{T}}\\
	\line(1,0){450}\\
		
	\noindent\textbf{\huge{U}}\\
	\line(1,0){450}\\
		
	\noindent\textbf{\huge{V}}\\
	\line(1,0){450}\\
	\textbf{VRaptor}: Framework para desenvolvimento Web para Java que utiliza 
	as ideias DRY e MVC.\\
	
	\noindent\textbf{\huge{W}}\\
	\line(1,0){450}\\
		
	\noindent\textbf{\huge{X}}\\
	\line(1,0){450}\\
	\textbf{XHTML}: eXtensible Hypertext Markup Language - é uma reformulação 
	da linguagem de marcação HTML, baseada em XML.\\
	\textbf{XML}: eXtensible Markup Language - é uma recomendação da W3C para 
	gerar linguagens de marcação para necessidades especiais.\\
	
	\noindent\textbf{\huge{Y}}\\
	\line(1,0){450}\\
		
	\noindent\textbf{\huge{Z}}\\
	\line(1,0){450}\\
	
\end{document}
