\documentclass[11pt, a4paper]{article}

\usepackage[brazil]{babel}
\usepackage[utf8]{inputenc}
\usepackage[T1]{fontenc}
\usepackage[pdftex]{hyperref}
\usepackage{graphicx}
\usepackage{amsmath}
\usepackage{indentfirst}
\usepackage{fancyhdr}


% Formatação
\topmargin -1.5cm
\oddsidemargin -0.04cm
\evensidemargin -0.04cm
\textwidth 16.59cm
\textheight 21.94cm 
%\pagestyle{empty}                     % Sem numero de paginas
\pagestyle{fancy}                         %cabecalhos e rodares
\fancyhead[RO,RE]{\today}
\fancyhead[LO,LE]{MAC0332 - SI para grupos de pesquisa\\
	Especificação dos Requerimentos}
%\fancyfoot[LO,LE]{Confidential}
\fancyfoot[RO,RE] {\thepage}
\fancyfoot[CO,CE]{Grupo 3, 2011}


\parskip 7.2pt                        % Espaço entre paragrafos 7.2
%\renewcommand{\baselinestretch}{1.5} % Espaçamento entre linhas = 1.5
%\parindent 0pt

% Tirar hifenização
\hyphenpenalty = 5000
\tolerance = 1000
\sloppy

\title{MAC 0332\\
	Engenharia de Software\\
	SI para grupos de pesquisa\\
	Architecture Notebook}
\date{\today}

\begin{document}

	\maketitle
	\newpage
	
	\section{Propósito}
        Este documento descreve a filosofia, as decisões, as restrições, as
        justificativas, os elementos significativos e qualquer outro aspecto
        importante do sistema que dão forma ao seu design e à sua implementação.

    %
    %[Always address Sections 2 through 6 of this template. Other sections are
    % recommended, depending on the amount of novel architecture, the amount of
    % expected maintenance, the skills of the development team, and the
    % importance of other architectural concerns.]
    %
    \section{Objetivos arquiteturais e filosofia}
        %
        %[Describe the philosophy of the architecture. Identify issues that will
        % drive the philosophy, such as: Will the system be driven by complex
        % deployment concerns, adapting to legacy systems, or performance
        % issues? Does it need to be robust for long-term maintenance?
        %
        % Formulate a set of goals that the architecture needs to meet in its
        % structure and behavior. Identify critical issues that must be
        % addressed by the architecture, such as: Are there hardware
        % dependencies that should be isolated from the rest of the system? Does
        % the system need to function efficiently under unusual conditions?]
        %
        A arquitetura desse projeto visa ao desenvolvimento de um sistema
        organizado, durável e prático tanto para desenvolvedores quanto para
        usuários.

        Através de ferramentas que abstraem implementações de recursos
        normalmente complexos (como banco de dados e interfaces Web), podemos
        de certa forma limitar o escopo de desenvolvimento à parte da lógica do
        sistema.

        Teremos menos preocupações e o volume de código será bastante reduzido,
        facilitando sua documentação, verificação e manutentção.

        Logo, os principais objetivos arquiteturais serão garantir que:

        \begin{enumerate}
            \item Essas ferramentas sejam devidamente aplicadas no sistema.
            Isso é, que elas estejam corretamente instaladas e instanciadas,
            e que nosso sistema as use e as faça se comunicarem entre si de
            maneira bem sucedida e aproveitando ao máximo suas facilidades;

            \item A lógica do sistema esteja de acordo com os requerimentos do
            cliente, usando uma estrutura o mais simples possível de testar, uma
            vez que a maioria das complicações técnicas estará sendo coberta
            pelas ferramentas auxiliares.
        \end{enumerate}

    \section{Suposições e dependências}
        %
        %[List the assumptions and dependencies that drive architectural
        % decisions. This could include sensitive or critical areas,
        % dependencies on legacy interfaces, the skill and experience of the
        % team, the availability of important resources, and so forth]
        %
        \subsection{Dependências tecnológicas do projeto:}
            \begin{itemize}
                \item Um servidor Tomcat (versão ???) para hospedar nosso
                sistema;
                
                \item Um banco de dados MySQL para persistir os dados
                gerenciados pelo sistema;
                
                \item a JDK e o Eclupse (Enterprise Edition ou equivalente),
                para desenvolvermos nossos sitema usando Java para Web;
                
                \item Um navegador para visualizarmos a interface Web do nosso
                sistema;
                
                \item A biblioteca JDBC para comunicar nossa aplicação Java com
                o banco de dados;
                
                \item A implementação Hibernate da JPA para abstrair o banco de
                dados através de um mapeamento objeto-relacional, facilitando
                o desenvolvimento do sistema em Java, que é uma linguagem
                orientada a objetos;
                
                \item A bilbioteca VRaptor, para comunicarmos facilmente nossa
                aplicação Java com a interface Web do sistema;
                
                \item As bibliotecas Junit e Mockito para testarmos nosso
                sistema;
                
                \item Git para controle de versão do projeto;
                
                \item Quaisquer outras dependências indiretas geradas pelas
                tecnologias citadas.
            \end{itemize}

        \subsection{Dependências de conhecimento e experiência da equipe:}
            \begin{itemize}
                \item Programar em Java usando o Eclipse;
                
                \item Desenvolver com orientação a objetos;
                
                \item Produzir páginas Web em HTML (com CSS);
                
                \item Modelar bancos de dados;
                
                \item Saber usar as bibliotecas Hibernate, VRaptor, JUnit e
                Mockito;
                
                \item Saber usar o controle de versão Git.
            \end{itemize}

        \subsection{Suposições:}
            \begin{itemize}
                \item Assumimos que as ferramentas, bibliotecas e demais
                tecnologias usadas no desenvolvimento do projeto são funcionais,
                isto é, agem do modo como suas respectivas documentações ditam
                e, portanto, não precisam ser testadas.
            \end{itemize}

    \section{Requisitos para realizar a arquitetura}
        %
        %[Insert a reference or link to the requirements that must be
        % implemented to realize the architecture.]
        %
        Para que se possa implementar a arquitetura descrita neste documento, a
        equipe de desenvolvimento deve ter os seguintes recursos devidamente
        instalados e funcionando:

        \begin{itemize}
            \item Ambiente de desenvolvimento Java usando Eclipse Enterprise
            Edition (ou equivalente);
            
            \item Um servidor Tomcat configurado para rodar a partir do Eclipse;
            
            \item Um repositório Git onde serão mantidos o código fonte do
            projeto e os diversos documentos importantes que forem produzidos ao
            longo do projeto (inclusive este);
            
            \item Um projeto em branco de VRaptor do site da Caelum importado no
            Eclipse. O aplicativo Java que conterá a lógica do nosso sistema
            será feito em cima dessa base. Esse projeto deve ser adicionado ao
            repositório Git, de preferência usando o \textit{plug-in} Git do
            Eclipse.
            Uma vez feito isso, todos da equipe poderão se sincronizar com o
            repositório para se unir ao desenvolvimento do sistema (ou seja, só
            uma pessoa precisa fazer isso e depois todos poderão importar esse
            projeto do Git usando o Eclipse);
            
            \item Os JAR's das bibliotecas JDBC, Hibernate (com annotations),
            JUnit (versão 4) e Mockito, bem como suas respectivas dependências,
            incluídas na pasta WebContent/WEB-INF/lib do projeto Java.
            A biblioteca VRaptor já vem incluída no projeto em branco da Caelum.
            (Do mesmo modo que no item anterior, uma vez que alguém tenha
            conseguido fazer isso, é só usar o repositório Git para que todos
            já tenham suas cópias locais do projeto igualmente configuradas.)
            %
            % Falta alguma coisa?
            %
        \end{itemize}

    \section{Decisões, restrições e justificativas}
        %
        %[List the decisions that have been made regarding architectural
        % approaches and the constraints being placed on the way that the
        % developers build the system. These will serve as guidelines for
        % defining architecturally significant parts of the system. Justify each
        % decision or constraint so that developers understand the importance of
        % building the system according to the context created by those
        % decisions and constraints. This may include a list of DOs and DON’Ts
        % to guide the developers in building the system.] 
        %
        \begin{itemize}
            \item O sistema será programado em Java, pois é uma linguagem
            orientada a objetos que possui diversas facilidades e recursos que
            auxiliam tanto no desenvolvimento quanto na manutenção do sistema.
            Além disso, há na comunidade diversas ferramentas disponíveis que
            facilitam a aplicação de Java na construção de sistemas de software
            pertinentes às exigências do mercado e dos círculos de pesquisa.
            
            \item O ambiente de desenvolvimento será o Eclipse Enterprise
            Edition (ou equivalente), pois ele vem com todas as ferramentas
            básicas para se programas para Web usando Java. Em particular, ele
            oferece a possibilidade de rodar um servidor Tomcat local no próprio
            Eclipse, o que possibilita que os desenvolvedores possam testar o
            sistema em suas próprias máquinas a qualquer momento, como se o
            estivessem rodando no servidor verdadeiro.

            \item Para comunicar nossa aplicação Java com a interface Web do
            sistema, usaremos a bilioteca VRaptor. Ela simplifica drasticamente
            a construção de controladores, além de gerenciar automaticamente as
            instâncias das diversas partes do sistema, criando-as e mantendo-as
            disponíveis conforme exigirem as demandas enviadas pelo cliente Web.

            \item Para comunicar nossa aplicação Java com o banco de dados,
            usaremos a biblioteca Hibernate, que se comunicará, por sua vez, com
            a biblioteca JDBC. Devido ao mapeamento objeto-relacional
            proporcionado pelo Hibernate (de acordo com a JPA), não haverá
            necessidade de nos preocuparmos com a criação manual das
            \textit{queries} ao banco de dados, pois isso será feito
            automaticamente. Pode-se programar quase que naturalmente usando
            orientação a objetos, onde classes corresponderão, aproximadamente,
            a relações.

            \item Para realizar testes de unidade nas nossas classes, usaremos a
            biblioteca Mockito. a JUnit é a ferramenta padrão de testes em Java,
            e tem \textit{plug-ins} no Eclipse que auxiliam seu uso. A Mockito
            servirá para que possamos simular objetos, impondo-lhes um
            comportamento pré-definido e, assim, podermos realizar testes de
            unidade que de fato não dependam de outras unidades para serem
            testados.
            %
            % \item [Decision or constraint and justification]
            % 
            % Wilson: "Adicionarei outras decisões conforme elas forem sendo
            % feitas. Por enquanto deixei apenas aquilo que já está decidido
            % pelas exigências do professor sobre o projeto."
            %
        \end{itemize}
    
    \section{Mecanismos arquiteturais}
        %
        %[List the architectural mechanisms and describe the current state of
        % each one. Initially, each mechanism may be only name and a brief
        % description. They will evolve until the mechanism is a collaboration
        % or pattern that can be directly applied to some aspect of the design.]
        %
        Essa é uma das partes que evoluirão ao longo do projeto, moldando-se de
        acordo com as necessidades que forem surgindo.

        \subsection{Controlador}
            Módulo responsável por controlar a parte lógica por trás da
            interface Web.

            Propósito: %TODO

            Funcionamento: %TODO

        \subsection{\textit{Data Access Object}}
            Módulo responsável por fornecer acesso a um certo tipo de dado do
            banco de dados.

            Propósito: %TODO

            Funcionamento: %TODO

        \subsection{Injeção de Dependência}
            Método utilizado para lidar com dependências de uma classe com
            outra. Sempre que uma classe precisar de um objeto de outra,
            deverá dar-se preferÊncia ao uso de injeções de dependência.

            Propósito: %TODO

            Funcionamento: %TODO

        \subsection{\textit{Factories}}
            Módulo que facilita a criação de objetos com configurações
            pré-determinadas. Comoatível com injeção de dependências.

            Propósito: %TODO

            Funcionamento: %TODO

        \subsection{Testes de unidade}
            Método utilizado para testar unidades individuais do sistema de
            maneira isolada.

            Propósito: %TODO

            Funcionamento: %TODO
        
        %
        % \subsection{Architectural Mechanism}
        %   [Describe the purpose, attributes, and function of the architectural
        %    mechanism.]
        %

    \section{Abstrações chave}
        %
        %[List and briefly describe the key abstractions of the system. This
        % should be a relatively short list of the critical concepts that define
        % the system. The key abstractions will usually translate to the initial
        % analysis classes and important patterns.]
        %
        \begin{description}
            \item[Interface Web] conjunto de páginas web que compõem o WebSite
            do sistema e que podem ser visualizadas pelos usuários. Envia
            requisições para o sistema de acordo com as ações dos usuários e
            mostra ao usuário as informações que o sistema lhe forneceu em
            resposta.

            \item[Banco de dados] conjunto de dados armazenados em um servidor
            seguindo um modelo relacional (usando MySQL, no caso). Os dados
            armazenados dizem respeito aos grupos de pesquisa, usuários,
            publicações e demais informações que foram passadas ao sistema de
            grupos de pesquisa.

            \item[Lógica do sistema] aplicativo (Java) que gerencia a lógica do
            sistema, interagindo com os usuários através da Interface Web e
            persistindo os dados obtidos no Banco de Dados.
            %
            % Wilson: "Dá pra colocar mais algumas coisas mas começaria a ficar
            % repetitivo com a seção anterior. Coloco mesmo assim?"
            %
        \end{description}

    \section{Camadas da arquitetura}
        %
        %[Describe the architectural pattern that you will use or how the
        % architecture will be consistent and uniform. This could be a simple
        % reference to an existing or well-known architectural pattern, such as
        % the Layer framework, a reference to a high-level model of the
        % framework, or a description of how the major system components should
        % be put together.]
        %
        Há no total seis camadas na arquitetura do projeto. A primeira, e mais
        externa, é a interface Web, composta pelos diversos arquivos
        \texttt{*.jsp} que faremos, formando as diversas páginas Web que os
        usuários poderão ver ao acessar nosso sistema.
        A segunda camada é o VRaptor, que serve para ligar a primeira camada com
        a terceira, que são os controladores. Estes, por sua vez, se comunicam
        com a quarta camada que é a que compõe a lógica do sistema. É nela que
        será focada nossa atenção e trabalho para construir o sistema.

        A quinta camada será composta pelos \textit{Data Access Objects} (DAOs),
        e servirá para a quarta camada (lógica do sistema) se comunicar com a
        sexta e última que é o banco de dados em si, só que abstraído pelo
        mapeamento objeto-relacional do Hibernate.
        %
        % Obs.1 Wilson: "assim que puder colocarei uma imagem na próxima seção
        % que ilustrará bem essas camadas"
        %
        % Obs.2 Wilson: "se for necessário posso detalhar melhor od que cada
        % camada será composta" (Samuel: "acho que não")
        %

    \section{Visões arquiteturais}
        %
        %[Describe the architectural views that you will use to describe the
        % software architecture. This illustrates the different perspectives
        % that you will make available to review and to document architectural
        % decisions.]
        %
        % --- Recommended views ---
        %
        % *Logical: Describes the structure and behavior of architecturally
        % significant portions of the system. This might include the package
        % structure, critical interfaces, important classes and subsystems, and
        % the relationships between these elements. It also includes physical
        % and logical views of persistent data, if persistence will be built
        % into the system. This is a documented subset of the design.
        %
        % *Operational: Describes the physical nodes of the system and the
        % processes, threads, and components that run on those physical nodes.
        % This view isn’t necessary if the system runs in a single process and
        % thread.
        %
        % *Use case: A list or diagram of the use cases that contain
        % architecturally significant requirements.
        %
        %
        \begin{itemize}
            \item \textbf{Visão lógica:} %TODO
            
            \item \textbf{Visão das classes:} %TODO
            
            \item \textbf{Casos de uso com requisitos arquiteturalmente
            importantes:} %TODO
        \end{itemize}
\end{document}
