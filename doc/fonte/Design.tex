\documentclass[11pt, a4paper]{article}

\usepackage[brazil]{babel}
\usepackage[utf8]{inputenc}
\usepackage[T1]{fontenc}
\usepackage[pdftex]{hyperref}
\usepackage{graphicx}
\usepackage{amsmath}
\usepackage{indentfirst}
\usepackage{fancyhdr}


% Formatação
\topmargin -1.5cm
\oddsidemargin -0.04cm
\evensidemargin -0.04cm
\textwidth 16.59cm
\textheight 21.94cm 
%\pagestyle{empty}                     % Sem numero de paginas
\pagestyle{fancy}                         %cabecalhos e rodares
\fancyhead[RO,RE]{\today}
\fancyhead[LO,LE]{MAC0332 - SI para grupos de pesquisa\\
	Design}
\fancyfoot[LO,LE]{Confidential}
\fancyfoot[RO,RE] {\thepage}
\fancyfoot[CO,CE]{Grupo 3, 2011}


\parskip 7.2pt                        % Espaço entre paragrafos 7.2
%\renewcommand{\baselinestretch}{1.5} % Espaçamento entre linhas = 1.5
%\parindent 0pt

% Tirar hifenização
\hyphenpenalty = 5000
\tolerance = 1000
\sloppy

\title{MAC 0332\\
	Engenharia de Software\\
	SI para grupos de pesquisa\\
	Design}
\date{\today}

\begin{document}

	\maketitle
	\newpage
	
	\section{Estrutura do Design}
		%[Describe the design from the highest level. This is commonly done with a diagram that shows a layered architecture.]
	\section{Subsistemas}
        %[Describe the design of a portion of the system (a package or component, for instance). The design should capture both static and dynamic perspectives. When capturing dynamic descriptions of behavior, look for reusable chunks of behavior that you can reference to simplify the design of the requirement realizations. You can break this section down into lower-level subsections to describe lower-level, encapsulated subsystems.]
            \subsection{Subsistema 1}
            
            \subsection{Subsistema 2}
            
            \subsection{Subsistema....}

	\section{Padrões}
        \subsection{Padrão 1}
		
		    \subsubsection{Visão Global}
                %[Provide an overview of the pattern in words in some consistent form. The overview of a pattern can include the intent, motivation, and applicability.]
		
		    \subsubsection{Estrutura}
                %[Describe the pattern from a static perspective. Include all of the participants and how they relate to one another, and call out the relevant data and behavior.]
		
		    \subsubsection{Comportamento}
                %[Describe the pattern from a dynamic perspective. Walk the reader through how the participants collaborate to support various scenarios.]
		
		    \subsubsection{Exemplo}
		        %[Often, you can convey the nature of the pattern better with an additional concrete example.]
		        
		\subsection{Padrão 2}
		
		    \subsubsection{Visão Global}
                %[Provide an overview of the pattern in words in some consistent form. The overview of a pattern can include the intent, motivation, and applicability.]
		
		    \subsubsection{Estrutura}
                %[Describe the pattern from a static perspective. Include all of the participants and how they relate to one another, and call out the relevant data and behavior.]
		
		    \subsubsection{Comportamento}
                %[Describe the pattern from a dynamic perspective. Walk the reader through how the participants collaborate to support various scenarios.]
		
		    \subsubsection{Exemplo}
		        %[Often, you can convey the nature of the pattern better with an additional concrete example.]
		        
		\subsection{Padrão.....}		
		    ......

		
	\section{Realizações de Exigência}
        
        \subsection{Realização 1}
		
	        \subsubsection{Visão dos Participantes}
                %[Describe the participating design elements from a static perspective, giving details such as behavior, relationships, and attributes relevant to this realization.]
		
	        \subsubsection{Cenário Básico}
                %[For the main flow, describe how instances of the design elements collaborate to realize the requirements. When using UML, this can be done with collaboration diagrams (sequence or communication).]
					
	        \subsubsection{Cenários Adicionais}
                %[For other scenarios that must be described to convey an appropriate amount of information about how the requirement behavior will be realized, describe how instances of the design elements collaborate to realize the requirement. When using UML, you can do this with collaboration diagrams (sequence or communication).]
                
        \subsection{Realização 2}
		
	        \subsubsection{Visão dos Participantes}
                %[Describe the participating design elements from a static perspective, giving details such as behavior, relationships, and attributes relevant to this realization.]
		
	        \subsubsection{Cenário Básico}
                %[For the main flow, describe how instances of the design elements collaborate to realize the requirements. When using UML, this can be done with collaboration diagrams (sequence or communication).]
					
	        \subsubsection{Cenários Adicionais}
                %[For other scenarios that must be described to convey an appropriate amount of information about how the requirement behavior will be realized, describe how instances of the design elements collaborate to realize the requirement. When using UML, you can do this with collaboration diagrams (sequence or communication).]
                
        \subsection{Realização ...}
            .......

\end{document}
